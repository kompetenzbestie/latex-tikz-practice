\documentclass[10pt,fleqn, a4paper]{article}

\usepackage{amsfonts}
\usepackage{amsmath}
\usepackage{amsthm}
\usepackage[ngerman]{babel}
\usepackage[left=2cm, right=3cm]{geometry}
\usepackage{tabularx}
\usepackage{tikz}
\usepackage{amssymb}

\usetikzlibrary{shapes, positioning, automata}

\begin{document}

{\bf Gruppen und Ringe}\newline

\begin{center}

\begin{tabular}{|c|c|c|c|c|}\hline
&Assoziativ& Neut. Element&Inv. Element&Kommutativ\\\hline\hline
Halbgruppe&$\checkmark$&&&\\\hline
Monoid&$\checkmark$&$\checkmark$&&\\\hline
Gruppe&$\checkmark$&$\checkmark$&$\checkmark$&\\\hline
Abelsche Gr.&$\checkmark$&$\checkmark$&$\checkmark$&$\checkmark$\\\hline
\end{tabular}


\vspace{0.5cm}

\begin{itemize}
\item[(R1)] $(R,\oplus)$ ist abelsche Gruppe.
\item[(R2)] $(R,\odot)$ ist Halbgruppe.
\item[(R3)] $\forall a,b,c\in R:a\odot(b\oplus c)=(a\odot b)\oplus (a\odot c)$\\
$\forall a,b,c\in R:(a\oplus b)\odot c=(a\odot c)\oplus (b\odot c)$
\end{itemize}

$(R,\odot)$ kommutatives Monoid $\rightarrow$ Kommutativer Ring mit Einselement

$(R\setminus\{0\},\oplus)$ abelsche Gruppe $\rightarrow$ Körper

\end{center}

\vspace{1cm}

{\bf Komplexe Zahlen}

\begin{align*}
z&=x+iy,\ w=u+iv\\\\
z\pm w&:= (x\pm u)+i(y\pm v)\\
z\cdot w&:=(xu-yu)+i(xv+yu)\\
\frac{z}{w}&:=\frac{xu+yv}{u^2+v^2}+i\frac{yu-xv}{u^2+v^2}\text{ nur für }w\neq 0
\end{align*}

Polarkoordinaten $\rightarrow$ Kartesische Koordinaten $(a,\phi)\rightarrow z=a\ cos\phi+i a\ sin\phi$

Kartesische Koordinaten $\rightarrow$ Polarkoordinaten $z \rightarrow (|z|,\ arg\ z)$ mit $arg\ z\left\{\begin{matrix}
arccos\frac{x}{|z|}\text{ falls }y\geq0\\
-arccos\frac{x}{|z|}\text{ falls }y<0
\end{matrix}\right.$

\vspace{1cm}

{\bf Vektorraumaxiome}

\begin{enumerate}
\item $(V,\oplus)$ ist eine abelsche Gruppe	mit neutralem Element $\vec 0$
\item $\forall\lambda,\mu\in K\forall\vec v\in V:\lambda\odot(\mu\odot\vec v)=(\lambda\cdot\mu)\odot\vec v$
\item $\forall\vec v\in V:1\odot\vec v=\vec v$
\item $\forall\lambda,\mu\in K\forall\vec v\in V:(\lambda+\mu)\odot\vec v=(\lambda\odot\vec v)\oplus(\mu\odot\vec v)$
\item $\forall\lambda\in K\forall\vec v,\vec w\in V:\lambda\odot(\vec v\oplus\vec w)=(\lambda\odot\vec v)\oplus(\lambda\odot\vec w)$
\end{enumerate}

\vspace{1cm}

{\bf Lineare Abbildungen}

\begin{itemize}
\item Monomorphismus, wenn $f$ injektiv
\item Epimorphismus, wenn $f$ surjektiv
\item Isomorphismus, wenn $f$ bijektiv
\item Endomorphismus, wenn $V=W$
\item Automorphismus, wenn $V=W$ und $f$ bijektiv
\end{itemize}

\begin{center}
$dim\ V=dim(Ker\ f)+dim(Im\ f)=dim(Ker\ f)+rg\ f$
\end{center}

\vspace{1cm}

{\bf Basiswechsel}

\begin{center}
$A^{\mathcal B'}_{\mathcal C'}=T^{\mathcal C}_{\mathcal C'}\cdot A^{\mathcal B}_{\mathcal C}\cdot T^{\mathcal B'}_{\mathcal B}$
\end{center}

Wobei $T^{\mathcal B'}_{\mathcal B}$ die Darstellung von $\mathcal B$ in der neuen Basis $\mathcal B'$ ist.

\vspace{1cm}

{\bf Determinanten}

Verhalten von Determinanten bei elementaren Zeilenumformungen:

\begin{enumerate}
\item $det\ A'=-det\ A$
\item $det\ A'=\lambda\cdot det\ A$
\item Die Determinante verändert sich nicht.
\end{enumerate}

\vspace{1cm}

{\bf Kram mit Normen}

\begin{align*}
\|\vec v\|&\geq 0\\
\|\vec v\|&=0\Leftrightarrow\vec v=\vec 0\\
\|\lambda\vec v\|&=|\lambda|\cdot\|\vec v\|\\
\|\vec u+\vec v\|&\leq \|\vec v\|+\|\vec u\|\\
\sphericalangle (\vec u,\vec v)&=arccos\frac{\left<\vec u,\vec v\right>}{\|\vec u\|\|\vec v\|}
\end{align*}

\vspace{1cm}

{\bf Gram-Schmidtsches Orthonormalisierungsverfahren}

\begin{align*}
\tilde u_1&=\vec v_1 &\text{und }\vec u_1=\frac{\tilde u_1}{\|\tilde u_1\|}\\
\tilde u_2&=\vec v_2-\left<\vec v_2,\vec u_1\right>\cdot\vec u_1 &\text{und }\vec u_2=\frac{\tilde u_2}{\|\tilde u_2\|}\\
\tilde u_k&=\vec v_k-\sum^{k-1}_{i=1}\left<\vec v_k,\vec u_i\right>\cdot\vec u_i &\text{und }\vec u_k=\frac{\tilde u_k}{\|\tilde u_k\|}
\end{align*}

\vspace{1cm}

{\bf Arten von Codes}

\begin{align*}
C\text{ ist }k\text{-fehlererkennend}&\Leftrightarrow\forall\vec v\in C:B_k(\vec c)\cap C=\{\vec c\}\\
&\Leftrightarrow d(C)\geq k+1\\\\
C\text{ ist }k\text{-fehlerkorrigierend}&\Leftrightarrow\forall\vec c,\vec c'\in C:\left(\vec c\neq\vec c'\Rightarrow B_k(\vec c)\cap B_k(\vec c')=\emptyset\right)\\
&\Leftrightarrow d(C)\geq 2k+1
\end{align*}

\end{document}