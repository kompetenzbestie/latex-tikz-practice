\documentclass[10pt,fleqn]{article}

\usepackage{amsfonts}
\usepackage{amsmath}
\usepackage{amsthm}
\usepackage[ngerman]{babel}
\usepackage[left=3cm, right=3cm]{geometry}
\usepackage{tikz}

\begin{document}

\centering
\begin{tikzpicture}
\tikzstyle{every node}=[font=\footnotesize]

% OSI

\node [font=\large, anchor=west] at (0, 7.5) {OSI};

\draw (0, 6) rectangle node {Application Layer} (3.5, 7);
\draw (0, 5) rectangle node {Presentation Layer} (3.5, 6);
\draw (0, 4) rectangle node {Session Layer} (3.5, 5);
\draw (0, 3) rectangle node {Transport Layer} (3.5, 4);
\draw (0, 2) rectangle node {Network Layer} (3.5, 3);
\draw (0, 1) rectangle node {Data Link Layer} (3.5, 2);
\draw (0, 0) rectangle node {Physical Layer} (3.5, 1);


% TCP/IP

\node [font=\large, anchor=west] at (4, 7.5) {TCP/IP};

\node at (5.75, 6.5) {Application Layer};
\draw (4, 4) rectangle (7.5, 7);
\draw [dashed] (4, 6) -- (7.5, 6);
\draw [dashed] (4, 5) -- (7.5, 5);

\draw (4, 3) rectangle node {Transport Layer} (7.5, 4);
\draw (4, 2) rectangle node {Network Layer} (7.5, 3);
\draw (4, 0) rectangle node [align=center] {Network Access\\Layer} (7.5, 2);


% TCP/IP-Protocols

\node [font=\large, anchor=west] at (8, 7.5) {TCP/IP-Protocols};

\draw (8, 4) rectangle node [rotate=90] {HTTP} (9, 7);
\draw (9, 4) rectangle node [rotate=90] {SMTP} (10, 7);
\draw (10, 4) rectangle node [rotate=90] {FTP} (11, 7);
\draw (11, 4) rectangle node [rotate=90] {DNS} (12, 7);
\draw (12, 4) rectangle node [rotate=90] {Telnet} (13, 7);
\draw (8, 3) rectangle node {TCP} (10.5, 4);
\draw (10.5, 3) rectangle node {UDP} (13, 4);
\draw (8, 2) rectangle node {IP} (13, 3);
\draw (8, 2) rectangle node {ARP} (9, 2.5);
\draw (11, 2.5) rectangle node {IGMP} (12, 3);
\draw (12, 2.5) rectangle node {ICMP} (13, 3);
\draw (8, 0) rectangle node {Ethernet} (9.25,2);
\draw (9.25, 0) rectangle node [align=center] {Token\\Ring} (10.5,2);
\draw (10.5, 0) rectangle node {ATM} (11.75,2);
\draw (11.75, 0) rectangle node [align=center] {Frame\\Relay} (13,2);

\end{tikzpicture}

\end{document}
